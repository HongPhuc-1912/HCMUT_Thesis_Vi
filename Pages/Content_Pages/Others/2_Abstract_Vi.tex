\newpage
\OtherPage{TÓM TẮT ĐỒ ÁN}

Việt tích hợp trực tiếp các nguồn năng lượng tái tạo vào lưới điện hiện nay đang gặp nhiều thách thức do tính chất không liên tục và phân tán của các nguồn năng lượng trên. Do vậy, lưới điện siêu nhỏ (Microgrid) là một phương pháp tiếp cận đầy hứa hẹn để tập hợp các máy phát điện phân tán cục bộ (Distributed Generator - DG), cung cấp cho các phụ tải cục bộ cũng như trao đổi với lưới điện tiện ích như một đơn vị có thể kiểm soát và điều khiển.Chế độ phát điện cục bộ - tiêu thụ cục bộ này có thể tránh được việc truyền tải điện đường dài, do đó có thể mang lại hiệu quả cao hơn. Các DG có thể được kết nối với một bus DC chung thông qua bộ chuyển đổi điện để tạo thành một lưới điện siêu nhỏ một chiều (DC Microgrid). Mục đích kiểm soát là làm cho nhiều DG chia sẻ tải đúng cách cũng như duy trì sự ổn định của điện áp. Đồ án này thảo luận về mô hình hóa, phân tích và kiểm soát microgrid DC với nhiều DG để cải thiện hiệu suất của nó ở trạng thái ổn định và trạng thái động. \par
Mặc dù điều khiển chủ-tớ truyền thống có thể đạt được sự điều chỉnh điện áp tốt và chia sẻ tải bằng nhau, nhưng sự phụ thuộc của nó vào giao tiếp băng thông cao và đơn vị chính làm giảm đáng kể độ tin cậy của hệ thống. Ngược lại, điều khiển droop cung cấp một sơ đồ điều khiển phân tán mà không cần giao tiếp. Là một phương pháp lập trình trở kháng đầu ra, điều khiển điện áp và chia sẻ tải được thực hiện tự động theo trở kháng đầu ra của DG. Do đó, nó nhạy cảm với trở kháng cáp kết nối và độ lệch tham chiếu điện áp danh định trong các ứng dụng điện áp thấp. \par
Trong đồ án này, lưới điện siêu nhỏ DC và phương pháp điều khiển Droop truyền thống được xem xét và phân tích, từ đó đề ra những nhược điểm hiện có của phương pháp điều khiển này. Đồng thời, ở trạng thái ổn định, một phương pháp bù sử dụng tham chiếu dòng điện chung được đề xuất để tăng cường hiệu suất chia sẻ tải và điện áp bus DC đồng thời. Biên độ của hệ số bù điện áp được phân tích bằng cách sử dụng các thử nghiệm ổn định tín hiệu nhỏ. Các mô phỏng trong MATLAB/Simulink và các thử nghiệm thực nghiệm trong phòng thí nghiệm được thực hiện để xác minh hiệu quả của phương pháp được đề xuất. \par