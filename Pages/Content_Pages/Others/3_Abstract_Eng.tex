\newpage
\OtherPage{ABSTRACT}
Integrating renewable energy sources directly into the current power grid poses significant challenges due to their intermittent and decentralized nature. Therefore, the microgrid approach emerges as a promising solution to consolidate locally distributed generators (DGs), supplying local loads while interacting with the utility grid as a controllable and manageable unit. This local generation-consumption mode can eliminate the need for long-distance electricity transmission, thereby improving overall efficiency. DGs can be connected to a common DC bus via power converters to form a DC microgrid. The control objective is to ensure proper load sharing among DGs while maintaining voltage stability. This thesis discusses the modeling, analysis, and control of DC microgrids with multiple DGs to enhance their performance in both steady-state and dynamic conditions. \par
Although traditional master-slave control can achieve good voltage regulation and equal load sharing, its reliance on high-bandwidth communication and a central unit significantly reduces system reliability. In contrast, droop control provides a decentralized control scheme without requiring communication. As a method of programming output impedance, it enables automatic voltage regulation and load sharing according to the output impedance of DGs. However, it is sensitive to the impedance of connecting cables and reference voltage deviations in low-voltage applications. \par
In this project, DC microgrids and traditional droop control methods are reviewed and analyzed to identify existing drawbacks. Additionally, under steady-state conditions, a compensation method using a common current reference is proposed to enhance load-sharing performance and DC bus voltage simultaneously. The magnitude of the voltage compensation factor is analyzed through small-signal stability tests. Simulations in MATLAB/Simulink and experimental tests in the laboratory are conducted to verify the effectiveness of the proposed method. \par